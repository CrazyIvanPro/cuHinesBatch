\chapterimage{head1} % Chapter heading image

\chapter{Human Brain Project}

The Human Brain Project (HBP) was launched by the European Commission's Future and 
Emerging Technologies (FET) scheme in October 2013, and has been 
scheduled to run for ten years. The Flagships represent a new partnering
 model for visionary, long-term European cooperative research in the 
 European Research Area, demonstrating the potential for common research
  efforts. The HBP has the following main objectives:

\begin{itemize}
    \item Create and operate a European Scientific Research Infrastructure for brain research, cognitive neuroscience, and other brain-inspired sciences
    \item Gather, organise and disseminate data describing the brain and its diseases
    \item Simulate the brain
    \item Build multi-scale scaffold theory and models for the brain
    \item Develop brain-inspired computing, data analytics and robotics
    \item Ensure that the HBP's work is undertaken responsibly and that it benefits society.
\end{itemize}

The HBP is now in its final phase with its Vision and Mission focused on:

\paragraph{The HBP Vision}

To deepen understanding of human brain structure and function, by 
building a European research infrastructure that harnesses multiple disciplines and computing, and advances science, ICT and medicine, to the benefit of society.

\paragraph{The HBP Missions}

1) To explore the multi-level complexity of the brain in space and time.

2) To transfer the acquired knowledge to brain-derived applications in health, computing, and technology.

3) To provide shared, open computing tools, models and data through the European Brain Research Infrastructure “EBRAINS – Solutions for Neuroscience” that serves to integrate brain science across disciplines.

4) To create a trans-disciplinary community of researchers united by the quest to understand the brain at multiple scales of organisation and functioning and thus derive societal benefits.

\section{Human Brain Simulation}

Can you imagine a brain and its workings being replicated on a computer? 
That is what the Human Brain Simulation aims to do. 


Today, we can find multiple initiatives that attempt to simulate the 
behavior of the Human Brain by computer simulations [9, 5, 7]. This is 
one of the most important challenges in the recent history of computing
with a large number of practical applications. The main constraint 
is being able to simulate efficiently a huge number of neurons using 
the current computer technology. However, the challenge is a complex one,
as the human brain contains 86 billion neurons each with an average 
of 7,000 connections to other neurons (known as synapses). 
Current computer power is insufficient to model a entire human brain
at this level of interconnectedness.

\vspace{5ex}
\begin{figure}[htbp]
    \centering
    \includegraphics[width = 0.7\textwidth]{brainsimu}
    \label{fig:brainsimu}
    \caption{Human Brain Simulation}
\end{figure}

 
A simpler approach has thus been adopted to produce results that are 
increasingly close approximations to experimental data. Simulation
takes place at several separate organisational levels in the brain, 
ranging from the molecular though the subcellular to cellular and up 
to the whole organ. The level of detail decreases as the level rises
towards the whole organ.

At the microscopic level and below, the signalling between neurons is the
focus. Neurons are electrically excitable cells that transmit messages
to each other across the synapses. These messages are crucial to the 
normal functioning of the central nervous system. The macroscopic level
examines assemblies of neurons, and their roles within the brain.

One of the most efficient ways in which the scientific community attempts
to simulate the behavior of the Human Brain consists of computing the 
next 3 major steps [6]: The computing of 
\begin{itemize}
    \item 1) the Voltage on neuron morphology
    \item 2) the synaptic elements in each of the neurons
    \item 3) the connectivity between the neurons.
\end{itemize}

In this survey, we focus on the first step which is one of the most 
consuming time steps of the simulation. Also it is strongly linked with 
the rest of steps. All these steps must be carried out on each of the 
neurons. The Human Brain is composed by about 14 thousand million of 
neurons, which are completely different among them in size and shape.


\vspace{10ex}
\section{Arbor Simulator}

Current simulators were designed for single core systems, with parallel implementations
added later. There are efforts to add many core support to existing codes, however they
are subject to the law of diminishing returns, this means that adding more cores could
even cause a fall in performance. A common example is adding more people to a job,
such as the assembly of a car on a factory floor. At some point, adding more workers causes problems such as workers getting in each other’s way or frequently finding
themselves waiting for access to a part.

This presents an opportunity to start work on the next generation of simulators,
designed from the ground up to support diverse many core architectures. Led by ETH
Zurich, \cite{Arbor}, as one of the simulators born inside the philosophy
 of the Human Brain project, aims to fill this gap.

\vspace{5ex}
\begin{figure}[htbp]
    \centering
    \includegraphics[width = 0.5\textwidth]{fig-11}
    \label{fig:11}
    \caption{Arbor main behavior}
\end{figure}

The simulation itself is divided into two big tasks, communication and computation, exchange and update-cells respectively on \ref{fig:11}. On communication, each
simulated neuron sends the spikes generated on one simulation time step to the other
interconnected neurons, the way that we determined if a spike is generated or not is
through the computation phase. As we can see, the communication phase depends on
the results obtained by the computation phase, leading us to one of the key points of
Arbor implementation: each time step of the simulation is half step of the behavior
that we are simulating, allowing a temporal pipeline implementation.

\vspace{5ex}
\begin{figure}[htbp]
    \centering
    \includegraphics[width = 0.5\textwidth]{fig-12}
    \label{fig:12}
    \caption{Arbor simulation pipeline}
\end{figure}

Despite being a great optimization that enables a increment in the parallelism,
it is not the main focus of our thesis, that is the reason why we are going to shift
our attention into the computation phase, the one that is in charge of determining
the Voltage on neuron morphology and one of the most time consuming steps of the
simulation.



