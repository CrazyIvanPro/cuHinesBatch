%%%%%%%%%%%%%%%%%%%%%%%%%%%%%%%%%%%%%%%%%
%  My documentation report
%  Objective: Explain what I did and how, in order to help someone continue with the investigation
%
% Important note:
% Chapter heading images should have a 2:1 width:height ratio,
% e.g. 920px width and 460px height.
%
% The images can be found anywhere, usually on sky surveys websites or the
% Astronomy Picture of the day archive http://apod.nasa.gov/apod/archivepix.html
%
% The original template (the Legrand Orange Book Template) can be found here --> http://www.latextemplates.com/template/the-legrand-orange-book
%
% Original author of the Legrand Orange Book Template:
% Mathias Legrand (legrand.mathias@gmail.com) with modifications by:
% Vel (vel@latextemplates.com)
%
% Original License:
% CC BY-NC-SA 3.0 (http://creativecommons.org/licenses/by-nc-sa/3.0/)
%%%%%%%%%%%%%%%%%%%%%%%%%%%%%%%%%%%%%%%%%
 
%----------------------------------------------------------------------------------------
%	PACKAGES AND OTHER DOCUMENT CONFIGURATIONS
%----------------------------------------------------------------------------------------

\documentclass[11pt,fleqn]{book} % Default font size and left-justified equations

\usepackage[top=3cm,bottom=3cm,left=3.2cm,right=3.2cm,headsep=10pt,letterpaper]{geometry} % Page margins

\usepackage{xcolor} % Required for specifying colors by name
\definecolor{ocre}{RGB}{52,177,201} % Define the orange color used for highlighting throughout the book

% Font Settings
\usepackage{avant} % Use the Avantgarde font for headings
%\usepackage{times} % Use the Times font for headings
\usepackage{mathptmx} % Use the Adobe Times Roman as the default text font together with math symbols from the Sym­bol, Chancery and Com­puter Modern fonts

\usepackage{microtype} % Slightly tweak font spacing for aesthetics
\usepackage[utf8]{inputenc} % Required for including letters with accents
\usepackage[T1]{fontenc} % Use 8-bit encoding that has 256 glyphs
\usepackage[UTF8]{ctex}

% Bibliography
\usepackage[style=alphabetic,sorting=nyt,sortcites=true,autopunct=true,babel=hyphen,hyperref=true,abbreviate=false,backref=true,backend=biber]{biblatex}
\addbibresource{bibliography.bib} % BibTeX bibliography file
\defbibheading{bibempty}{}

\input{structure} % Insert the commands.tex file which contains the majority of the structure behind the template

\begin{document}
\title{Introduction to
Parallel & Distributed Computing}

%----------------------------------------------------------------------------------------
%	TITLE PAGE
%----------------------------------------------------------------------------------------

\begingroup
\thispagestyle{empty}
\AddToShipoutPicture*{\put(0,0){\includegraphics[scale=1.25]{esahubble}}} % Image background
\centering
\vspace*{5cm}
\par\normalfont\fontsize{32}{32}\sffamily\selectfont
\textbf{Introduction to
Parallel \& Distributed Computing}\\
\vspace*{1cm}
{\LARGE \textbf{A Survey on Human Brain Project and Neuromorphic Brain Simulation}}\par % Book title
\vspace*{8cm}
{\textcolor{white}{\Huge Yifan Zhang}}\par % Author name
\endgroup

%----------------------------------------------------------------------------------------
%	COPYRIGHT PAGE
%----------------------------------------------------------------------------------------

\newpage
~\vfill
\thispagestyle{empty}

%\noindent Copyright \copyright\ 2014 Andrea Hidalgo\\ % Copyright notice

\noindent \textsc{Peking University}\\

\noindent This research was done under the supervision of Prof. Luo.\\ % License information

\noindent \textit{First release, June 2020} % Printing/edition date

%----------------------------------------------------------------------------------------
%	TABLE OF CONTENTS
%----------------------------------------------------------------------------------------

\chapterimage{head0.png} % Table of contents heading image

\pagestyle{empty} % No headers

\tableofcontents % Print the table of contents itself

%\cleardoublepage % Forces the first chapter to start on an odd page so it's on the right

\pagestyle{fancy} % Print headers again

%------------------------------------------------

%----------------------------------------------------------------------------------------
%	CHAPTER 1
%----------------------------------------------------------------------------------------

\chapterimage{head1} % Chapter heading image

\chapter{Human Brain Project}

The Human Brain Project (HBP) was launched by the European Commission's Future and 
Emerging Technologies (FET) scheme in October 2013, and has been 
scheduled to run for ten years. The Flagships represent a new partnering
 model for visionary, long-term European cooperative research in the 
 European Research Area, demonstrating the potential for common research
  efforts. The HBP has the following main objectives:

\begin{itemize}
    \item Create and operate a European Scientific Research Infrastructure for brain research, cognitive neuroscience, and other brain-inspired sciences
    \item Gather, organise and disseminate data describing the brain and its diseases
    \item Simulate the brain
    \item Build multi-scale scaffold theory and models for the brain
    \item Develop brain-inspired computing, data analytics and robotics
    \item Ensure that the HBP's work is undertaken responsibly and that it benefits society.
\end{itemize}

The HBP is now in its final phase with its Vision and Mission focused on:

\paragraph{The HBP Vision}

To deepen understanding of human brain structure and function, by 
building a European research infrastructure that harnesses multiple disciplines and computing, and advances science, ICT and medicine, to the benefit of society.

\paragraph{The HBP Missions}

1) To explore the multi-level complexity of the brain in space and time.

2) To transfer the acquired knowledge to brain-derived applications in health, computing, and technology.

3) To provide shared, open computing tools, models and data through the European Brain Research Infrastructure “EBRAINS – Solutions for Neuroscience” that serves to integrate brain science across disciplines.

4) To create a trans-disciplinary community of researchers united by the quest to understand the brain at multiple scales of organisation and functioning and thus derive societal benefits.

\section{Human Brain Simulation}

Can you imagine a brain and its workings being replicated on a computer? 
That is what the Human Brain Simulation aims to do. 


Today, we can find multiple initiatives that attempt to simulate the 
behavior of the Human Brain by computer simulations [9, 5, 7]. This is 
one of the most important challenges in the recent history of computing
with a large number of practical applications. The main constraint 
is being able to simulate efficiently a huge number of neurons using 
the current computer technology. However, the challenge is a complex one,
as the human brain contains 86 billion neurons each with an average 
of 7,000 connections to other neurons (known as synapses). 
Current computer power is insufficient to model a entire human brain
at this level of interconnectedness.

\vspace{5ex}
\begin{figure}[htbp]
    \centering
    \includegraphics[width = 0.7\textwidth]{brainsimu}
    \label{fig:brainsimu}
    \caption{Human Brain Simulation}
\end{figure}

 
A simpler approach has thus been adopted to produce results that are 
increasingly close approximations to experimental data. Simulation
takes place at several separate organisational levels in the brain, 
ranging from the molecular though the subcellular to cellular and up 
to the whole organ. The level of detail decreases as the level rises
towards the whole organ.

At the microscopic level and below, the signalling between neurons is the
focus. Neurons are electrically excitable cells that transmit messages
to each other across the synapses. These messages are crucial to the 
normal functioning of the central nervous system. The macroscopic level
examines assemblies of neurons, and their roles within the brain.

One of the most efficient ways in which the scientific community attempts
to simulate the behavior of the Human Brain consists of computing the 
next 3 major steps [6]: The computing of 
\begin{itemize}
    \item 1) the Voltage on neuron morphology
    \item 2) the synaptic elements in each of the neurons
    \item 3) the connectivity between the neurons.
\end{itemize}

In this survey, we focus on the first step which is one of the most 
consuming time steps of the simulation. Also it is strongly linked with 
the rest of steps. All these steps must be carried out on each of the 
neurons. The Human Brain is composed by about 14 thousand million of 
neurons, which are completely different among them in size and shape.


\vspace{10ex}
\section{Arbor Simulator}

Current simulators were designed for single core systems, with parallel implementations
added later. There are efforts to add many core support to existing codes, however they
are subject to the law of diminishing returns, this means that adding more cores could
even cause a fall in performance. A common example is adding more people to a job,
such as the assembly of a car on a factory floor. At some point, adding more workers causes problems such as workers getting in each other’s way or frequently finding
themselves waiting for access to a part.

This presents an opportunity to start work on the next generation of simulators,
designed from the ground up to support diverse many core architectures. Led by ETH
Zurich, \cite{Arbor}, as one of the simulators born inside the philosophy
 of the Human Brain project, aims to fill this gap.

\vspace{5ex}
\begin{figure}[htbp]
    \centering
    \includegraphics[width = 0.5\textwidth]{fig-11}
    \label{fig:11}
    \caption{Arbor main behavior}
\end{figure}

The simulation itself is divided into two big tasks, communication and computation, exchange and update-cells respectively on \ref{fig:11}. On communication, each
simulated neuron sends the spikes generated on one simulation time step to the other
interconnected neurons, the way that we determined if a spike is generated or not is
through the computation phase. As we can see, the communication phase depends on
the results obtained by the computation phase, leading us to one of the key points of
Arbor implementation: each time step of the simulation is half step of the behavior
that we are simulating, allowing a temporal pipeline implementation.

\vspace{5ex}
\begin{figure}[htbp]
    \centering
    \includegraphics[width = 0.5\textwidth]{fig-12}
    \label{fig:12}
    \caption{Arbor simulation pipeline}
\end{figure}

Despite being a great optimization that enables a increment in the parallelism,
it is not the main focus of our thesis, that is the reason why we are going to shift
our attention into the computation phase, the one that is in charge of determining
the Voltage on neuron morphology and one of the most time consuming steps of the
simulation.



 % Finished

%----------------------------------------------------------------------------------------
%	CHAPTER 2
%----------------------------------------------------------------------------------------

\chapterimage{head2} % Chapter heading image

\chapter{TODO}



%----------------------------------------------------------------------------------------
%	CHAPTER 3
%----------------------------------------------------------------------------------------

\chapterimage{head3} % Chapter heading image

\chapter{Experimental Results}

Here is the Makefile for this project:

\begin{lstlisting}[title = {Makefile}]
    CC		   = g++
NVCC       = nvcc
LD_LIBRARY_PATH= /usr/local/cuda/lib64
CUDA_LIB   = -L /usr/local/cuda/lib64/ -lcuda -lcudart
 -lcusparse -lcusolver 
ARCH_CUDA  = -arch=sm_30
NVCC_FLAGS = --ptxas-options=-v -Xcompiler -fopenmp -O3 
-std=c++11  -D_MWAITXINTRIN_H_INCLUDED -D_FORCE_INLINES  
GCC_FLAGS  = -fopenmp -std=c++11 -Wall -pedantic  


all: clean serial parallel remove

serial: memm.o cuThomasVBatch.o serial.cc
    $(CC)   $(GCC_FLAGS) -o serial serial.cc $(CUDA_LIB) 
    memm.o cuThomasVBatch.o	

parallel: memm.o cuThomasVBatch.o parallel.cc
    $(CC)   $(GCC_FLAGS) -o parallel parallel.cc $(CUDA_LIB) 
    memm.o cuThomasVBatch.o	

memm.o: memm.cu

	$(NVCC) -c  $(NVCC_FLAGS) $(ARCH_CUDA) memm.cu

cuThomasVBatch.o: cuThomasVBatch.cu 
    $(NVCC) -c  $(NVCC_FLAGS) $(ARCH_CUDA) 
    cuThomasVBatch.cu	 

clean:
	rm -rf *.o serial parallel

remove:
	rm -rf *.o
\end{lstlisting}

\vspace{5ex}
Here is the experimental results:

\begin{table}[htbp]
	\caption{Compare the metrics of different implementations (Case 8)}
	\centering
	\begin{tabular}[width=1.0\linewidth]{lllllll}
		\toprule
		\quad Method & Case 7 & Case 8 & Case 9 & Case 10 & Case 11 & Case 12\\
    \midrule
    Serial            & 1.832603e-03 & 4.852763e-03 & 1.556245e-2  & 2.745925e-2  & 1.34865e-1   & 1.624341e-1   \\
    OpenMP            & 6.644011e-02 & 2.541431e-03 & 3.947942e-2  & 2.346893e-2  & 5.79814e-1   & 3.565425e-1   \\
    cuHinesBatch      & 8.718967e-03 & 2.536453e-03 & 1.746943e-2  & 1.914897e-2  & 3.31285e-2   & 4.681494e-2   \\
    \bottomrule
	\end{tabular}
	\label{tab:table1}
\end{table}

\vspace{1ex}

From the results above, we could find out that the speedup > 1 for Multi-CPU and GPU implemenations both in large cases.
On the consideration of the parallel overhead, when the numbers $N$ rises, the paralleled version have better performance compared to the serial one.


\vspace{10ex}

Besides, we also tested the Heterogeneous Parallelization, but the performance is not satisfying.



\chapterimage{head-xappendix}
\chapter{Conclusions}

large-scale brain simulation is not the only
plausible way to approach the problem of building human-like
artificial general intelligence, but it's certainly a sensible candidate approach. So far, the goals of brain simulation projects have
mainly to do with emulating the general properties of brain
subsystems, rather than creating integrated systems (or even
subsystems) that carry out intelligent functions analogously to
the human brain. But once more progress has been made this may
shift, and the line between whole-brain simulations and artificial
intelligence systems may blur. Whether this will be the first
approach to lead to success at human-like AGI remains unclear,
but it is certainly a plausible and promising approach, and
Markram and other simulation researchers have their long-term
sights set specifically in this direction.

The achievements of these benefits is still speculative, and a
great amount of work remains to be done, in brain simulation but
also in other areas such as refinement of the underlying neural
modeling equations, the gathering of neurophysiological data, and
the provision of adequately scalable (and economical) computing
hardware. However we see little doubt that large-scale brain
simulation has a great amount to contribute in the above areas
and others.

%----------------------------------------------------------------------------------------
%	REFERENCES
%----------------------------------------------------------------------------------------

\clearpage
\newpage
\chapterimage{head-ref} % Table of contents heading image
\begin{thebibliography}{1}
  \bibitem{cuHines}
  Pedro Valero-Lara, and Ivan Martinez-Perez.
  cuHinesBatch: Solving Multiple Hines systems on GPUs Human Brain Project. 
  International Conference on Computational 
  Science, 2017, 12-14

  \bibitem{Arbor}
  N. A. Akar, B. Cumming, V. Karakasis, A. Kusters, W. Klijn, A. Peyser and S. Yates,
  Arbor --- A Morphologically-Detailed Neural Network Simulation Library for Contemporary High-Performance Computing Architectures,

  \bibitem{Roy13}
  Roy Ben-Shalom, Gilad Liberman, and Alon Korngreen. Accelerating compartmental modeling
  on a graphical processing unit. Frontiers in Neuroanatomy, 7:4, 2013.

  \bibitem{samue80}
  Samuel Daniel Conte and Carl W. De Boor. Elementary Numerical Analysis: An Algorithmic
  Approach. McGraw-Hill Higher Education, 3rd edition, 1980.

  \bibitem{cuda}
  cuSPARSE. Nvidia-cuda toolkit documentation. http://docs.nvidia.com/cuda/cusparse/.

  \bibitem{andrew}
  Andrew A. Davidson, Yao Zhang, and John D. Owens. An auto-tuned method for solving large
tridiagonal systems on the GPU. In 25th IEEE International Symposium on Parallel and Distributed
  Processing, IPDPS, Anchorage, Alaska, USA, pages 956–965, May 2011.
  
  \bibitem{HBP}
  Ecole Polytechnique Federale de Lausanne (EPFL). The Blue Brain Project.
  http://bluebrain.epfl.ch/.

  \bibitem{Stochastic}
  S. Karlin, and H. M. Taylor. A First Course in Stochastic Processes. Academic Press, 1975.

  \bibitem{Sandra16}
  Sandra Diaz-Pier, Mikal Naveau, Markus Butz-Ostendorf, and Abigail Morrison. Automatic generation
  of connectivity for large-scale neuronal network models through structural plasticity. Frontiers
  in Neuroanatomy, 10:57, 2016.

  \bibitem{pedro16}
  Pedro Valero-Lara, Poornima Nookala, Fernando L. Pelayo, Johan Jansson, Serapheim Dimitropoulos,
  and Ioan Raicu. Many-task computing on many-core architectures. Scalable Computing:
  Practice and Experience, 17(1):32–46, 2016.

    \bibitem{chapter2-1}
  M. Gell-Mann and F. E. Low. Quantum Electrodynamics at Small Distances. Phys. Rev. 95, 1300, 1954

  \bibitem{foundation}
  Jennings Nicholas R, and Wooldridge Michael J. Foundations of Machine Learning. Foundations of machine learning. MIT Press, 2012.

  \bibitem{sanjeev18}
  Sanjeev A., Nadav C., and    Elad H. On the Optimization of Deep Networks: Implicit Acceleration by Overparameterization. In 35th International Conference on Machine Learning, 2018.
  \bibitem{huishuai19}
  Huishuai Z., Da Y., Mingyang Y., Wei C., and Tie-yan L. Convergence Theory of Learning Over-parameterized ResNet: A Full Characterization. arXiv preprint arXiv:1903.07120, 2019.
  \bibitem{yuanzhi18}
  Yuanzhi L. and Yingyu L. Learning Overparameterized Neural Networks via Stochastic Gradient Descent on Structured Data. InAdvances in Neural Information Processing Systems, pages 8157–8166, 2018.
  \bibitem{difan18}
  Difan Z., Yuan C., Dongruo Z., and Quanquan G. Stochastic Gradient Descent Optimizes Over-parameterized Deep ReLU Networks. arXiv preprint arXiv:1811.08888, 2018.
  \bibitem{jonathan18}
  Jonathan F., and Michael C. The Lottery Ticket Hypotheis: Finding Sparse, Trainable Neural Networks. In7th International Conference on Learning Representations, ICLR, 2019.
  \bibitem{li17}
  Li, Y., Ma, T., \& Zhang, H. Algorithmic regularization in over-parameterized matrix sensing and neural networks with quadratic activations. arXiv preprint arXiv:1712.09203, 2017.
  \bibitem{du18}
  Du, S. S., Hu, W., \& Lee, J. D. Algorithmic regularization in learning deep homogeneous models: Layers are automatically balanced. In Advances in Neural Information Processing Systems (pp. 384-395).
  
  \bibitem{Harold73}
  Harold S. Stone. An efficient parallel algorithm for the solution of a tridiagonal linear system of
  equations. J. ACM, 20(1):27–38, January 1973.
  
  \bibitem{pedro14}
  Pedro Valero-Lara. Multi-gpu acceleration of DARTEL (early detection of alzheimer). In 2014
  IEEE International Conference on Cluster Computing, CLUSTER 2014, Madrid, Spain, September
  22-26, 2014, pages 346–354, 2014.
  
  \bibitem{chen98}
  S.S. Chen, D.L. Donoho, M.A. Saunders. Atomic Decomposition by Basis Pursuit. SIAM Journal on Scientific Computing 20(1), p.33-61, 1998.
  \bibitem{jure20}
  Jure Leskovec, Anand Rajaraman, Jeff Ullman. Mining of Massive Datasets. Cambridge Press, 2020.
  % 1.10
  \bibitem{srinivas10}
  Srinivas, N., et. al. Gaussian process optimization in the bandit setting: No regret and experimental design. ICML 2010.
  \bibitem{jones98}
  Jones, D., et. al., Efficient global optimization of expensive black-box functions. J. Global Optimization, 1998.

  \bibitem{cuSPARSE}
  cuSPARSE. Nvidia-cuda toolkit documentation.
  http://docs.nvidia.com/cuda/cusparse/

  \bibitem{andrew11}
  Andrew Davidson, Yao Zhang, and John D. Owens. An auto-tuned method for
solving large tridiagonal systems on the GPU. In IEEE International Parallel and
Distributed Processing Symposium, May 2011.

  \bibitem{sakharnykh10}
  N. Sakharnykh. Efficient tridiagonal solvers for adi methods and fluid simulation.
  In NVIDIA GPU Technology Conference, September 2010.

  \bibitem{pedro12}
  Pedro Valero-Lara, Alfredo Pinelli, Julien Favier, and Manuel Prieto Matias. Block
tridiagonal solvers on heterogeneous architectures. In IEEE 10th International
Symposium on Parallel and Distributed Processing with Applications, ISPA ’12,
pages 609–616, 2012.
  \end{thebibliography}

%----------------------------------------------------------------------------------------
%	APPENDIX
%----------------------------------------------------------------------------------------

\clearpage
\newpage
\chapterimage{head-xappendix.png} % Table of contents heading image
\appendix
  \renewcommand{\appendixname}{Appendix~\Alph{section}}
    \chapter{Appendix}
    \section[Acknowledgement]{Acknowledgement}
    感谢党政军各级领导, 国家机关以及各级行政机构, 面对 2020 新冠疫情所做出的重要决策与战略部署.

    \vspace{5ex}
    感谢全国人民, 各级社会组织和基层单位, 在抗击 2020 新冠疫情过程中所做出的奉献与牺牲.
    
    \vspace{5ex}
    感谢北京大学, 罗国杰老师, 课程助教以及各位教职员工为本学期线上教学科研工作的顺利开展所做出的努力.

  % \chapter*{List of Theorems and Definitions}
  % \listoftheorems%[ignoreall,show=definition]




% \textit{Wish you all the best, Yifan Zhang}
\end{document}

