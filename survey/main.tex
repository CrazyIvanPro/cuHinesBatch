%%%%%%%%%%%%%%%%%%%%%%%%%%%%%%%%%%%%%%%%%
%  My documentation report
%  Objective: Explain what I did and how, in order to help someone continue with the investigation
%
% Important note:
% Chapter heading images should have a 2:1 width:height ratio,
% e.g. 920px width and 460px height.
%
% The images can be found anywhere, usually on sky surveys websites or the
% Astronomy Picture of the day archive http://apod.nasa.gov/apod/archivepix.html
%
% The original template (the Legrand Orange Book Template) can be found here --> http://www.latextemplates.com/template/the-legrand-orange-book
%
% Original author of the Legrand Orange Book Template:
% Mathias Legrand (legrand.mathias@gmail.com) with modifications by:
% Vel (vel@latextemplates.com)
%
% Original License:
% CC BY-NC-SA 3.0 (http://creativecommons.org/licenses/by-nc-sa/3.0/)
%%%%%%%%%%%%%%%%%%%%%%%%%%%%%%%%%%%%%%%%%
 
%----------------------------------------------------------------------------------------
%	PACKAGES AND OTHER DOCUMENT CONFIGURATIONS
%----------------------------------------------------------------------------------------

\documentclass[11pt,fleqn]{book} % Default font size and left-justified equations

\usepackage[top=3cm,bottom=3cm,left=3.2cm,right=3.2cm,headsep=10pt,letterpaper]{geometry} % Page margins

\usepackage{xcolor} % Required for specifying colors by name
\definecolor{ocre}{RGB}{52,177,201} % Define the orange color used for highlighting throughout the book

% Font Settings
\usepackage{avant} % Use the Avantgarde font for headings
%\usepackage{times} % Use the Times font for headings
\usepackage{mathptmx} % Use the Adobe Times Roman as the default text font together with math symbols from the Sym­bol, Chancery and Com­puter Modern fonts

\usepackage{microtype} % Slightly tweak font spacing for aesthetics
\usepackage[utf8]{inputenc} % Required for including letters with accents
\usepackage[T1]{fontenc} % Use 8-bit encoding that has 256 glyphs
\usepackage[UTF8]{ctex}

% Bibliography
\usepackage[style=alphabetic,sorting=nyt,sortcites=true,autopunct=true,babel=hyphen,hyperref=true,abbreviate=false,backref=true,backend=biber]{biblatex}
\addbibresource{bibliography.bib} % BibTeX bibliography file
\defbibheading{bibempty}{}

%%%%%%%%%%%%%%%%%%%%%%%%%%%%%%%%%%%%%%%%%
% This is based on the Legrand Orange Book
% Structural Definitions File
%
% The original template (the Legrand Orange Book Template) can be found here --> http://www.latextemplates.com/template/the-legrand-orange-book
%
% Original author of the Legrand Orange Book Template::
% Mathias Legrand (legrand.mathias@gmail.com) with modifications by:
% Vel (vel@latextemplates.com)
%
% Original License:
% CC BY-NC-SA 3.0 (http://creativecommons.org/licenses/by-nc-sa/3.0/)
%
%%%%%%%%%%%%%%%%%%%%%%%%%%%%%%%%%%%%%%%%%
%----------------------------------------------------------------------------------------
%	VARIOUS REQUIRED PACKAGES
%----------------------------------------------------------------------------------------

\usepackage{titlesec} % Allows customization of titles

\usepackage{color} % Allows customization of font color

\usepackage{graphicx} % Required for including pictures
\graphicspath{{Pictures/}} % Specifies the directory where pictures are stored

\usepackage{lipsum} % Inserts dummy text

\usepackage{tikz} % Required for drawing custom shapes

\usepackage[english]{babel} % English language/hyphenation

\usepackage{enumitem} % Customize lists
\setlist{nolistsep} % Reduce spacing between bullet points and numbered lists

\usepackage{booktabs} % Required for nicer horizontal rules in tables

\usepackage{eso-pic} % Required for specifying an image background in the title page

\usepackage{hyperref}       % hyperlinks
\usepackage{url}            % simple URL typesetting
\usepackage{booktabs}       % professional-quality tables
\usepackage{amsfonts}       % blackboard math symbols
\usepackage{nicefrac}       % compact symbols for 1/2, etc.
\usepackage{microtype}      % microtypography
\usepackage{lipsum}
\usepackage[linesnumbered,boxed,ruled,commentsnumbered]{algorithm2e}
\usepackage{algorithmic}
\usepackage{amsmath}
\usepackage{amsthm,thmtools}
\usepackage[nottoc]{tocbibind}
\usepackage{nccmath}
\usepackage{amssymb}
\usepackage{amstext}
\usepackage{subfigure}
\usepackage{graphicx}
\usepackage{listings}



\lstset{
  language = C++, numbers=left, 
      numberstyle=\tiny,keywordstyle=\color{blue!70},
      commentstyle=\color{red!50!green!50!blue!50},frame=shadowbox,
      rulesepcolor=\color{red!20!green!20!blue!20},basicstyle=\ttfamily
}

%----------------------------------------------------------------------------------------
%	MAIN TABLE OF CONTENTS
%----------------------------------------------------------------------------------------

\usepackage{titletoc} % Required for manipulating the table of contents

\contentsmargin{0cm} % Removes the default margin
% Chapter text styling
\titlecontents{chapter}[1.25cm] % Indentation
{\addvspace{15pt}\large\sffamily\bfseries} % Spacing and font options for chapters
{\color{ocre!60}\contentslabel[\Large\thecontentslabel]{1.25cm}\color{ocre}} % Chapter number
{}  
{\color{ocre!60}\normalsize\sffamily\bfseries\;\titlerule*[.5pc]{.}\;\thecontentspage} % Page number
% Section text styling
\titlecontents{section}[1.25cm] % Indentation
{\addvspace{5pt}\sffamily\bfseries} % Spacing and font options for sections
{\contentslabel[\thecontentslabel]{1.25cm}} % Section number
{}
{\sffamily\hfill\color{black}\thecontentspage} % Page number
[]
% Subsection text styling
\titlecontents{subsection}[1.25cm] % Indentation
{\addvspace{1pt}\sffamily\small} % Spacing and font options for subsections
{\contentslabel[\thecontentslabel]{1.25cm}} % Subsection number
{}
{\sffamily\;\titlerule*[.5pc]{.}\;\thecontentspage} % Page number
[] 

%----------------------------------------------------------------------------------------
%	MINI TABLE OF CONTENTS IN CHAPTER HEADS
%----------------------------------------------------------------------------------------

% Section text styling
\titlecontents{lsection}[0em] % Indendating
{\footnotesize\sffamily} % Font settings
{}
{}
{}

% Subsection text styling
\titlecontents{lsubsection}[.5em] % Indentation
{\normalfont\footnotesize\sffamily} % Font settings
{}
{}
{}
 
%----------------------------------------------------------------------------------------
%	PAGE HEADERS
%----------------------------------------------------------------------------------------

\usepackage{fancyhdr} % Required for header and footer configuration

\pagestyle{fancy}
\renewcommand{\chaptermark}[1]{\markboth{\sffamily\normalsize\bfseries\chaptername\ \thechapter.\ #1}{}} % Chapter text font settings
\renewcommand{\sectionmark}[1]{\markright{\sffamily\normalsize\thesection\hspace{5pt}#1}{}} % Section text font settings
\fancyhf{} \fancyhead[LE,RO]{\sffamily\normalsize\thepage} % Font setting for the page number in the header
\fancyhead[LO]{\rightmark} % Print the nearest section name on the left side of odd pages
\fancyhead[RE]{\leftmark} % Print the current chapter name on the right side of even pages
\renewcommand{\headrulewidth}{0.5pt} % Width of the rule under the header
\addtolength{\headheight}{2.5pt} % Increase the spacing around the header slightly
\renewcommand{\footrulewidth}{0pt} % Removes the rule in the footer
\fancypagestyle{plain}{\fancyhead{}\renewcommand{\headrulewidth}{0pt}} % Style for when a plain pagestyle is specified

% Removes the header from odd empty pages at the end of chapters
\makeatletter
\renewcommand{\cleardoublepage}{
\clearpage\ifodd\c@page\else
\hbox{}
\vspace*{\fill}
\thispagestyle{empty}
\newpage
\fi}

%----------------------------------------------------------------------------------------
%	THEOREM STYLES
%----------------------------------------------------------------------------------------

\usepackage{amsmath,amsfonts,amssymb,amsthm} % For math equations, theorems, symbols, etc

\newcommand{\intoo}[2]{\mathopen{]}#1\,;#2\mathclose{[}}
\newcommand{\ud}{\mathop{\mathrm{{}d}}\mathopen{}}
\newcommand{\intff}[2]{\mathopen{[}#1\,;#2\mathclose{]}}
\newtheorem{notation}{Notation}[chapter]

%%%%%%%%%%%%%%%%%%%%%%%%%%%%%%%%%%%%%%%%%%%%%%%%%%%%%%%%%%%%%%%%%%%%%%%%%%%
%%%%%%%%%%%%%%%%%%%% dedicated to boxed/framed environements %%%%%%%%%%%%%%
%%%%%%%%%%%%%%%%%%%%%%%%%%%%%%%%%%%%%%%%%%%%%%%%%%%%%%%%%%%%%%%%%%%%%%%%%%%
\newtheoremstyle{ocrenumbox}% % Theorem style name
{0pt}% Space above
{0pt}% Space below
{\normalfont}% % Body font
{}% Indent amount
{\small\bf\sffamily\color{ocre}}% % Theorem head font
{\;}% Punctuation after theorem head
{0.25em}% Space after theorem head
{\small\sffamily\color{ocre}\thmname{#1}\nobreakspace\thmnumber{\@ifnotempty{#1}{}\@upn{#2}}% Theorem text (e.g. Theorem 2.1)
\thmnote{\nobreakspace\the\thm@notefont\sffamily\bfseries\color{black}---\nobreakspace#3.}} % Optional theorem note
\renewcommand{\qedsymbol}{$\blacksquare$}% Optional qed square

\newtheoremstyle{blacknumex}% Theorem style name
{5pt}% Space above
{5pt}% Space below
{\normalfont}% Body font
{} % Indent amount
{\small\bf\sffamily}% Theorem head font
{\;}% Punctuation after theorem head
{0.25em}% Space after theorem head
{\small\sffamily{\tiny\ensuremath{\blacksquare}}\nobreakspace\thmname{#1}\nobreakspace\thmnumber{\@ifnotempty{#1}{}\@upn{#2}}% Theorem text (e.g. Theorem 2.1)
\thmnote{\nobreakspace\the\thm@notefont\sffamily\bfseries---\nobreakspace#3.}}% Optional theorem note

\newtheoremstyle{blacknumbox} % Theorem style name
{0pt}% Space above
{0pt}% Space below
{\normalfont}% Body font
{}% Indent amount
{\small\bf\sffamily}% Theorem head font
{\;}% Punctuation after theorem head
{0.25em}% Space after theorem head
{\small\sffamily\thmname{#1}\nobreakspace\thmnumber{\@ifnotempty{#1}{}\@upn{#2}}% Theorem text (e.g. Theorem 2.1)
\thmnote{\nobreakspace\the\thm@notefont\sffamily\bfseries---\nobreakspace#3.}}% Optional theorem note

%%%%%%%%%%%%%%%%%%%%%%%%%%%%%%%%%%%%%%%%%%%%%%%%%%%%%%%%%%%%%%%%%%%%%%%%%%%
%%%%%%%%%%%%% dedicated to non-boxed/non-framed environements %%%%%%%%%%%%%
%%%%%%%%%%%%%%%%%%%%%%%%%%%%%%%%%%%%%%%%%%%%%%%%%%%%%%%%%%%%%%%%%%%%%%%%%%%
\newtheoremstyle{ocrenum}% % Theorem style name
{5pt}% Space above
{5pt}% Space below
{\normalfont}% % Body font
{}% Indent amount
{\small\bf\sffamily\color{ocre}}% % Theorem head font
{\;}% Punctuation after theorem head
{0.25em}% Space after theorem head
{\small\sffamily\color{ocre}\thmname{#1}\nobreakspace\thmnumber{\@ifnotempty{#1}{}\@upn{#2}}% Theorem text (e.g. Theorem 2.1)
\thmnote{\nobreakspace\the\thm@notefont\sffamily\bfseries\color{black}---\nobreakspace#3.}} % Optional theorem note
\renewcommand{\qedsymbol}{$\blacksquare$}% Optional qed square
\makeatother

% Defines the theorem text style for each type of theorem to one of the three styles above
\newcounter{dummy} 
\numberwithin{dummy}{section}
\theoremstyle{ocrenumbox}
\newtheorem{theoremeT}[dummy]{Theorem}
\newtheorem{problem}{Problem}[chapter]
\newtheorem{exerciseT}{Exercise}[chapter]
\theoremstyle{blacknumex}
\newtheorem{exampleT}{Example}[chapter]
\theoremstyle{blacknumbox}
\newtheorem{vocabulary}{Vocabulary}[chapter]
\newtheorem{definitionT}{Definition}[section]
\newtheorem{corollaryT}[dummy]{Corollary}
\theoremstyle{ocrenum}
\newtheorem{proposition}[dummy]{Proposition}

%----------------------------------------------------------------------------------------
%	DEFINITION OF COLORED BOXES
%----------------------------------------------------------------------------------------

\RequirePackage[framemethod=default]{mdframed} % Required for creating the theorem, definition, exercise and corollary boxes

% Theorem box
\newmdenv[skipabove=15pt,
skipbelow=7pt,
backgroundcolor=black!5,
linecolor=ocre,
innerleftmargin=5pt,
innerrightmargin=5pt,
innertopmargin=5pt,
leftmargin=0cm,
rightmargin=0cm,
innerbottommargin=5pt]{tBox}

% Exercise box	  
\newmdenv[skipabove=15pt,
skipbelow=7pt,
rightline=false,
leftline=true,
topline=false,
bottomline=false,
backgroundcolor=ocre!10,
linecolor=ocre,
innerleftmargin=5pt,
innerrightmargin=5pt,
innertopmargin=5pt,
innerbottommargin=5pt,
leftmargin=0cm,
rightmargin=0cm,
linewidth=4pt]{eBox}	

% Definition box
\newmdenv[skipabove=15pt,
skipbelow=7pt,
rightline=false,
leftline=true,
topline=false,
bottomline=false,
linecolor=ocre,
innerleftmargin=5pt,
innerrightmargin=5pt,
innertopmargin=0pt,
leftmargin=0cm,
rightmargin=0cm,
linewidth=4pt,
innerbottommargin=0pt]{dBox}	

% Corollary box
\newmdenv[skipabove=15pt,
skipbelow=7pt,
rightline=false,
leftline=true,
topline=false,
bottomline=false,
linecolor=gray,
backgroundcolor=black!5,
innerleftmargin=5pt,
innerrightmargin=5pt,
innertopmargin=5pt,
leftmargin=0cm,
rightmargin=0cm,
linewidth=4pt,
innerbottommargin=5pt]{cBox}

% Creates an environment for each type of theorem and assigns it a theorem text style from the "Theorem Styles" section above and a colored box from above
\newenvironment{theorem}{\begin{tBox}\begin{theoremeT}}{\end{theoremeT}\end{tBox}}
\newenvironment{exercise}{\begin{eBox}\begin{exerciseT}}{\hfill{\color{ocre}\tiny\ensuremath{\blacksquare}}\end{exerciseT}\end{eBox}}				  
\newenvironment{definition}{\begin{dBox}\begin{definitionT}}{\end{definitionT}\end{dBox}}	
\newenvironment{example}{\begin{exampleT}}{\hfill{\tiny\ensuremath{\blacksquare}}\end{exampleT}}		
\newenvironment{corollary}{\begin{cBox}\begin{corollaryT}}{\end{corollaryT}\end{cBox}}	

%----------------------------------------------------------------------------------------
%	REMARK ENVIRONMENT
%----------------------------------------------------------------------------------------

\newenvironment{remark}{\par\vspace{10pt}\small % Vertical white space above the remark and smaller font size
\begin{list}{}{
\leftmargin=35pt % Indentation on the left
\rightmargin=25pt}\item\ignorespaces % Indentation on the right
\makebox[-2.5pt]{\begin{tikzpicture}[overlay]
\node[draw=ocre!60,line width=1pt,circle,fill=ocre!25,font=\sffamily\bfseries,inner sep=2pt,outer sep=0pt] at (-15pt,0pt){\textcolor{ocre}{R}};\end{tikzpicture}} % Orange R in a circle
\advance\baselineskip -1pt}{\end{list}\vskip5pt} % Tighter line spacing and white space after remark

%----------------------------------------------------------------------------------------
%	SECTION NUMBERING IN THE MARGIN
%----------------------------------------------------------------------------------------

\makeatletter
\renewcommand{\@seccntformat}[1]{\llap{\textcolor{ocre}{\csname the#1\endcsname}\hspace{1em}}}                    
\renewcommand{\section}{\vspace{5ex}\@startsection{section}{1}{\z@}
{-4ex \@plus -1ex \@minus -.4ex}
{1ex \@plus.2ex }
{\normalfont\large\sffamily\bfseries}}
\renewcommand{\subsection}{\vspace{1ex}\@startsection {subsection}{2}{\z@}
{-3ex \@plus -0.1ex \@minus -.4ex}
{0.5ex \@plus.2ex }
{\normalfont\sffamily\bfseries}}
\renewcommand{\subsubsection}{\@startsection {subsubsection}{3}{\z@}
{-2ex \@plus -0.1ex \@minus -.2ex}
{.2ex \@plus.2ex }
{\normalfont\small\sffamily\bfseries}}                        
\renewcommand\paragraph{\@startsection{paragraph}{4}{\z@}
{-2ex \@plus-.2ex \@minus .2ex}
{.1ex}
{\normalfont\small\sffamily\bfseries}}

%----------------------------------------------------------------------------------------
%	HYPERLINKS IN THE DOCUMENTS
%----------------------------------------------------------------------------------------

% For an unclear reason, the package should be loaded now and not later
\usepackage{hyperref}
\hypersetup{hidelinks,backref=true,pagebackref=true,hyperindex=true,colorlinks=false,breaklinks=true,urlcolor= ocre,bookmarks=true,bookmarksopen=false,pdftitle={Title},pdfauthor={Author}}

%----------------------------------------------------------------------------------------
%	CHAPTER HEADINGS
%----------------------------------------------------------------------------------------

% The set-up below should be (sadly) manually adapted to the overall margin page septup controlled by the geometry package loaded in the main.tex document. It is possible to implement below the dimensions used in the goemetry package (top,bottom,left,right)... TO BE DONE

\newcommand{\thechapterimage}{}
\newcommand{\chapterimage}[1]{\renewcommand{\thechapterimage}{#1}}

% Numbered chapters with mini tableofcontents
\def\thechapter{\arabic{chapter}}
\def\@makechapterhead#1{
\thispagestyle{empty}
{\centering \normalfont\sffamily
\ifnum \c@secnumdepth >\m@ne
\if@mainmatter
\startcontents
\begin{tikzpicture}[remember picture,overlay]
\node at (current page.north west)
{\begin{tikzpicture}[remember picture,overlay]
\node[anchor=north west,inner sep=0pt] at (0,0) {\includegraphics[width=\paperwidth]{\thechapterimage}};
%%%%%%%%%%%%%%%%%%%%%%%%%%%%%%%%%%%%%%%%%%%%%%%%%%%%%%%%%%%%%%%%%%%%%%%%%%%%%%%%%%%%%
% Commenting the 3 lines below removes the small contents box in the chapter heading
%\fill[color=ocre!10!white,opacity=.6] (1cm,0) rectangle (8cm,-7cm);
%\node[anchor=north west] at (1.1cm,.35cm) {\parbox[t][8cm][t]{6.5cm}{\huge\bfseries\flushleft \printcontents{l}{1}{\setcounter{tocdepth}{2}}}};
\draw[anchor=west] (5cm,-9cm) node [rounded corners=20pt,fill=ocre!10!white,text opacity=1,draw=ocre,draw opacity=1,line width=1.5pt,fill opacity=.6,inner sep=12pt]{\huge\sffamily\bfseries\textcolor{black}{\thechapter. #1\strut\makebox[22cm]{}}};
%%%%%%%%%%%%%%%%%%%%%%%%%%%%%%%%%%%%%%%%%%%%%%%%%%%%%%%%%%%%%%%%%%%%%%%%%%%%%%%%%%%%%
\end{tikzpicture}};
\end{tikzpicture}}
\par\vspace*{230\p@}
\fi
\fi}

% Unnumbered chapters without mini tableofcontents (could be added though) 
\def\@makeschapterhead#1{
\thispagestyle{empty}
{\centering \normalfont\sffamily
\ifnum \c@secnumdepth >\m@ne
\if@mainmatter
\begin{tikzpicture}[remember picture,overlay]
\node at (current page.north west)
{\begin{tikzpicture}[remember picture,overlay]
\node[anchor=north west,inner sep=0pt] at (0,0) {\includegraphics[width=\paperwidth]{\thechapterimage}};
\draw[anchor=west] (5cm,-9cm) node [rounded corners=20pt,fill=ocre!10!white,fill opacity=.6,inner sep=12pt,text opacity=1,draw=ocre,draw opacity=1,line width=1.5pt]{\huge\sffamily\bfseries\textcolor{black}{#1\strut\makebox[22cm]{}}};
\end{tikzpicture}};
\end{tikzpicture}}
\par\vspace*{230\p@}
\fi
\fi
}
\makeatother

 % Insert the commands.tex file which contains the majority of the structure behind the template

\begin{document}
\title{Introduction to
Parallel & Distributed Computing}

%----------------------------------------------------------------------------------------
%	TITLE PAGE
%----------------------------------------------------------------------------------------

\begingroup
\thispagestyle{empty}
\AddToShipoutPicture*{\put(0,0){\includegraphics[scale=1.25]{esahubble}}} % Image background
\centering
\vspace*{5cm}
\par\normalfont\fontsize{32}{32}\sffamily\selectfont
\textbf{Introduction to
Parallel \& Distributed Computing}\\
\vspace*{1cm}
{\LARGE \textbf{A Survey on Human Brain Project and Neuromorphic Brain Simulation}}\par % Book title
\vspace*{8cm}
{\textcolor{white}{\Huge Yifan Zhang}}\par % Author name
\endgroup

%----------------------------------------------------------------------------------------
%	COPYRIGHT PAGE
%----------------------------------------------------------------------------------------

\newpage
~\vfill
\thispagestyle{empty}

%\noindent Copyright \copyright\ 2014 Andrea Hidalgo\\ % Copyright notice

\noindent \textsc{Peking University}\\

\noindent This research was done under the supervision of Prof. Luo.\\ % License information

\noindent \textit{First release, June 2020} % Printing/edition date

%----------------------------------------------------------------------------------------
%	TABLE OF CONTENTS
%----------------------------------------------------------------------------------------

\chapterimage{head0.png} % Table of contents heading image

\pagestyle{empty} % No headers

\tableofcontents % Print the table of contents itself

%\cleardoublepage % Forces the first chapter to start on an odd page so it's on the right

\pagestyle{fancy} % Print headers again

%------------------------------------------------

%----------------------------------------------------------------------------------------
%	CHAPTER 1
%----------------------------------------------------------------------------------------

\chapterimage{head1} % Chapter heading image

\chapter{TODO}

 % Finished

%----------------------------------------------------------------------------------------
%	CHAPTER 2
%----------------------------------------------------------------------------------------

\chapterimage{head2} % Chapter heading image

\chapter{TODO}



%----------------------------------------------------------------------------------------
%	CHAPTER 3
%----------------------------------------------------------------------------------------

\chapterimage{head3} % Chapter heading image

\chapter{Experimental Results}

Here is the Makefile for this project:

\begin{lstlisting}[title = {Makefile}]
    CC		   = g++
NVCC       = nvcc
LD_LIBRARY_PATH= /usr/local/cuda/lib64
CUDA_LIB   = -L /usr/local/cuda/lib64/ -lcuda -lcudart
 -lcusparse -lcusolver 
ARCH_CUDA  = -arch=sm_30
NVCC_FLAGS = --ptxas-options=-v -Xcompiler -fopenmp -O3 
-std=c++11  -D_MWAITXINTRIN_H_INCLUDED -D_FORCE_INLINES  
GCC_FLAGS  = -fopenmp -std=c++11 -Wall -pedantic  


all: clean serial parallel remove

serial: memm.o cuThomasVBatch.o serial.cc
    $(CC)   $(GCC_FLAGS) -o serial serial.cc $(CUDA_LIB) 
    memm.o cuThomasVBatch.o	

parallel: memm.o cuThomasVBatch.o parallel.cc
    $(CC)   $(GCC_FLAGS) -o parallel parallel.cc $(CUDA_LIB) 
    memm.o cuThomasVBatch.o	

memm.o: memm.cu

	$(NVCC) -c  $(NVCC_FLAGS) $(ARCH_CUDA) memm.cu

cuThomasVBatch.o: cuThomasVBatch.cu 
    $(NVCC) -c  $(NVCC_FLAGS) $(ARCH_CUDA) 
    cuThomasVBatch.cu	 

clean:
	rm -rf *.o serial parallel

remove:
	rm -rf *.o
\end{lstlisting}

\vspace{5ex}
Here is the experimental results:

\begin{table}[htbp]
	\caption{Compare the metrics of different implementations (Case 8)}
	\centering
	\begin{tabular}[width=1.0\linewidth]{lllllll}
		\toprule
		\quad Method & Case 7 & Case 8 & Case 9 & Case 10 & Case 11 & Case 12\\
    \midrule
    Serial            & 1.832603e-03 & 4.852763e-03 & 1.556245e-2  & 2.745925e-2  & 1.34865e-1   & 1.624341e-1   \\
    OpenMP            & 6.644011e-02 & 2.541431e-03 & 3.947942e-2  & 2.346893e-2  & 5.79814e-1   & 3.565425e-1   \\
    cuHinesBatch      & 8.718967e-03 & 2.536453e-03 & 1.746943e-2  & 1.914897e-2  & 3.31285e-2   & 4.681494e-2   \\
    \bottomrule
	\end{tabular}
	\label{tab:table1}
\end{table}

\vspace{1ex}

From the results above, we could find out that the speedup > 1 for Multi-CPU and GPU implemenations both in large cases.
On the consideration of the parallel overhead, when the numbers $N$ rises, the paralleled version have better performance compared to the serial one.


\vspace{10ex}

Besides, we also tested the Heterogeneous Parallelization, but the performance is not satisfying.



%----------------------------------------------------------------------------------------
%	REFERENCES
%----------------------------------------------------------------------------------------

\clearpage
\newpage
\chapterimage{head-ref} % Table of contents heading image
\begin{thebibliography}{1}
  \bibitem{cuHines}
  Pedro Valero-Lara, and Ivan Martinez-Perez.
  cuHinesBatch: Solving Multiple Hines systems on GPUs Human Brain Project. 
  International Conference on Computational 
  Science, 2017, 12-14

  \bibitem{Arbor}
  N. A. Akar, B. Cumming, V. Karakasis, A. Kusters, W. Klijn, A. Peyser and S. Yates,
  Arbor --- A Morphologically-Detailed Neural Network Simulation Library for Contemporary High-Performance Computing Architectures,

  \bibitem{Roy13}
  Roy Ben-Shalom, Gilad Liberman, and Alon Korngreen. Accelerating compartmental modeling
  on a graphical processing unit. Frontiers in Neuroanatomy, 7:4, 2013.

  \bibitem{samue80}
  Samuel Daniel Conte and Carl W. De Boor. Elementary Numerical Analysis: An Algorithmic
  Approach. McGraw-Hill Higher Education, 3rd edition, 1980.

  \bibitem{cuda}
  cuSPARSE. Nvidia-cuda toolkit documentation. http://docs.nvidia.com/cuda/cusparse/.

  \bibitem{andrew}
  Andrew A. Davidson, Yao Zhang, and John D. Owens. An auto-tuned method for solving large
tridiagonal systems on the GPU. In 25th IEEE International Symposium on Parallel and Distributed
  Processing, IPDPS, Anchorage, Alaska, USA, pages 956–965, May 2011.
  
  \bibitem{HBP}
  Ecole Polytechnique Federale de Lausanne (EPFL). The Blue Brain Project.
  http://bluebrain.epfl.ch/.

  \bibitem{Stochastic}
  S. Karlin, and H. M. Taylor. A First Course in Stochastic Processes. Academic Press, 1975.

  \bibitem{Sandra16}
  Sandra Diaz-Pier, Mikal Naveau, Markus Butz-Ostendorf, and Abigail Morrison. Automatic generation
  of connectivity for large-scale neuronal network models through structural plasticity. Frontiers
  in Neuroanatomy, 10:57, 2016.

  \bibitem{pedro16}
  Pedro Valero-Lara, Poornima Nookala, Fernando L. Pelayo, Johan Jansson, Serapheim Dimitropoulos,
  and Ioan Raicu. Many-task computing on many-core architectures. Scalable Computing:
  Practice and Experience, 17(1):32–46, 2016.

    \bibitem{chapter2-1}
  M. Gell-Mann and F. E. Low. Quantum Electrodynamics at Small Distances. Phys. Rev. 95, 1300, 1954

  \bibitem{foundation}
  Jennings Nicholas R, and Wooldridge Michael J. Foundations of Machine Learning. Foundations of machine learning. MIT Press, 2012.

  \bibitem{sanjeev18}
  Sanjeev A., Nadav C., and    Elad H. On the Optimization of Deep Networks: Implicit Acceleration by Overparameterization. In 35th International Conference on Machine Learning, 2018.
  \bibitem{huishuai19}
  Huishuai Z., Da Y., Mingyang Y., Wei C., and Tie-yan L. Convergence Theory of Learning Over-parameterized ResNet: A Full Characterization. arXiv preprint arXiv:1903.07120, 2019.
  \bibitem{yuanzhi18}
  Yuanzhi L. and Yingyu L. Learning Overparameterized Neural Networks via Stochastic Gradient Descent on Structured Data. InAdvances in Neural Information Processing Systems, pages 8157–8166, 2018.
  \bibitem{difan18}
  Difan Z., Yuan C., Dongruo Z., and Quanquan G. Stochastic Gradient Descent Optimizes Over-parameterized Deep ReLU Networks. arXiv preprint arXiv:1811.08888, 2018.
  \bibitem{jonathan18}
  Jonathan F., and Michael C. The Lottery Ticket Hypotheis: Finding Sparse, Trainable Neural Networks. In7th International Conference on Learning Representations, ICLR, 2019.
  \bibitem{li17}
  Li, Y., Ma, T., \& Zhang, H. Algorithmic regularization in over-parameterized matrix sensing and neural networks with quadratic activations. arXiv preprint arXiv:1712.09203, 2017.
  \bibitem{du18}
  Du, S. S., Hu, W., \& Lee, J. D. Algorithmic regularization in learning deep homogeneous models: Layers are automatically balanced. In Advances in Neural Information Processing Systems (pp. 384-395).
  
  \bibitem{Harold73}
  Harold S. Stone. An efficient parallel algorithm for the solution of a tridiagonal linear system of
  equations. J. ACM, 20(1):27–38, January 1973.
  
  \bibitem{pedro14}
  Pedro Valero-Lara. Multi-gpu acceleration of DARTEL (early detection of alzheimer). In 2014
  IEEE International Conference on Cluster Computing, CLUSTER 2014, Madrid, Spain, September
  22-26, 2014, pages 346–354, 2014.
  
  \bibitem{chen98}
  S.S. Chen, D.L. Donoho, M.A. Saunders. Atomic Decomposition by Basis Pursuit. SIAM Journal on Scientific Computing 20(1), p.33-61, 1998.
  \bibitem{jure20}
  Jure Leskovec, Anand Rajaraman, Jeff Ullman. Mining of Massive Datasets. Cambridge Press, 2020.
  % 1.10
  \bibitem{srinivas10}
  Srinivas, N., et. al. Gaussian process optimization in the bandit setting: No regret and experimental design. ICML 2010.
  \bibitem{jones98}
  Jones, D., et. al., Efficient global optimization of expensive black-box functions. J. Global Optimization, 1998.

  \bibitem{cuSPARSE}
  cuSPARSE. Nvidia-cuda toolkit documentation.
  http://docs.nvidia.com/cuda/cusparse/

  \bibitem{andrew11}
  Andrew Davidson, Yao Zhang, and John D. Owens. An auto-tuned method for
solving large tridiagonal systems on the GPU. In IEEE International Parallel and
Distributed Processing Symposium, May 2011.

  \bibitem{sakharnykh10}
  N. Sakharnykh. Efficient tridiagonal solvers for adi methods and fluid simulation.
  In NVIDIA GPU Technology Conference, September 2010.

  \bibitem{pedro12}
  Pedro Valero-Lara, Alfredo Pinelli, Julien Favier, and Manuel Prieto Matias. Block
tridiagonal solvers on heterogeneous architectures. In IEEE 10th International
Symposium on Parallel and Distributed Processing with Applications, ISPA ’12,
pages 609–616, 2012.
  \end{thebibliography}

%----------------------------------------------------------------------------------------
%	APPENDIX
%----------------------------------------------------------------------------------------

\clearpage
\newpage
\chapterimage{head-xappendix.png} % Table of contents heading image
\appendix
  \renewcommand{\appendixname}{Appendix~\Alph{section}}
    \chapter{Appendix}
    \section[Acknowledgement]{Acknowledgement}
    感谢党政军各级领导, 国家机关以及各级行政机构, 面对 2020 新冠疫情所做出的重要决策与战略部署.

    \vspace{5ex}
    感谢全国人民, 各级社会组织和基层单位, 在抗击 2020 新冠疫情过程中所做出的奉献与牺牲.
    
    \vspace{5ex}
    感谢北京大学, 罗国杰老师, 课程助教以及各位教职员工为本学期线上教学科研工作的顺利开展所做出的努力.

  % \chapter*{List of Theorems and Definitions}
  % \listoftheorems%[ignoreall,show=definition]




% \textit{Wish you all the best, Yifan Zhang}
\end{document}

