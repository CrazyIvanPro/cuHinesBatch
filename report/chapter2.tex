\chapterimage{head2} % Chapter heading image

\chapter{Implementation Details}

\section{Problem Formulation}

对于一棵具有 $N$ 个结点的树 $T$,根结点编号为 0,自树根至叶的编号呈深度优先次序,
每个结点具有以下六种属性(括号内为缩写):

\begin{itemize}
    \item index(id)
    \item upper(u)
    \item lower(l)
    \item diagonal(d)
    \item right-side-hand(rhs)
    \item parent(p)  
\end{itemize}

\vspace{1ex}
其中 parent 代表的父结点的 id。

\vspace{5ex}
该问题 包含两个阶段: 

\begin{itemize}
    \item 第一阶段完成由树叶至树根的一次遍历,记作 backward sweep;
    \item 第二阶段完成由树根至叶的一次遍历,记作 forward sweep;
\end{itemize}

\vspace{1ex}
两个阶段的次序不可交换。算法伪代码如下:

\vspace{5ex}
\begin{figure}[htbp]
    \centering
    \includegraphics[width = 0.8\textwidth]{description}
    \label{fig:description}
\end{figure}


\vspace{10ex}
\section{Implementation of cuThomasBatch}
For conveience, we start from the implementation of cuThomasBatch, which can seen as the original method of cuHinesBatch.

An efficient memory management is critical to achieve a good performance, but even
much more on those architectures based on a high throughput and a high memory
latency, such as the GPUs. In this sense, first we focus on presenting the different data
layouts proposed and analyze the impact of these on the overall performance. Two
different data layouts were explored: Flat and Full-Interleaved. While the Flat data
layout consists of storing all the elements of each of the systems in contiguous memory
locations, in the Full-Interleaved data layout, first, we store the first elements of each
of the systems in contiguous memory locations, after that we store the set of the second
elements, and so on until the last element.

\begin{figure}[htbp]
    \centering
    \includegraphics[width = 0.8\textwidth]{fig-31}
    \caption{Example of the \textit{Flat} data layout}
    \label{fig:fig-31}
\end{figure}

For sake of clarity, Figure \ref{fig:fig-31} and \ref{fig:fig-32} illustrate a simple example composed by four
different tridiagonal systems of three elements each. Please, note that we only illustrate
one vector per system in Figure 3.1, but in the real scenario we would have 4 vectors
per tridiagonal system on which are carried out the strategies above described. As
widely known, one of the most important requirements to achieve a good performance
on NVIDIA GPUs is to have contiguous threads accessing contiguous memory locations
(coalescing memory accesses). This is the main motivation behind the proposal of the
different data layouts and CUDA thread mappings. As later shown, the differences
found in the data layouts studied have important consequences on the scalability.

\begin{figure}[htbp]
    \centering
    \includegraphics[width = 0.8\textwidth]{fig-32}
    \caption{Example of the \textit{Flat} data layout}
    \label{fig:fig-32}
\end{figure}

Additionally, we have explored other data-layout, Unified-Vector. In this case, we
attempt to analyze the hierarchy of memory by exploiting the temporal locality that
there is among the different vectors. Basically, every
thread, immediately after $a_i$ is computed, has to compute also 
$b_i$ and $c_i$, in the forward step. In the backward step, the process
is similar, but in the opposite order. Using this data layout, we want to take advantage of
this characteristic of the \textit{Thomas} algorithm.
take advantage of this characteristic of the Thomas algorithm.
For sake of clarity, Figure \ref{fig:fig-33} 
graphically illustrates the data layout proposed for a simple batch 
composed by only 2 independent tridiagonal systems.

\begin{figure}[htbp]
    \centering
    \includegraphics[width = 0.8\textwidth]{fig-33}
    \caption{Example of the Unified-Vector data layout}
    \label{fig:fig-33}
\end{figure}

Next, we explore the different proposals about the CUDA thread mapping on the
data layouts above described. Figure \ref{fig:fig-34} illustrates the different CUDA thread mappings studied in this paper. Figure 3.4(top) shows a coarse-grain scheme where a set of
tridiagonal ($S_1$, $\cdots$, $S_n$ in Figure \ref{fig:fig-34}) systems
 is mapped onto a CUDA block, so that each CUDA thread fully solves one system. We decided to explore this approach to avoid
 dealing with atomic accesses and synchronizations, as well as to be able to execute a
 very high number of tridiagonal systems of any size, without the limitation imposed by
 the parallel methods.

 Using the Flat data layout we can not exploit coalescence when exploting one thread per tridiagonal
 system (coarse approach); however by interleaving (\ref{fig:fig-32}) the elements of the 
 vectors, contiguous threads access to contiguous memory locations. This approach does not
 exploit efficiently the shared memory of the GPUs, since the memory required by each
 CUDA thread becomes too large. Our GPU implementation (cuThomasBatch) is based on this approach,
 Thomas algorithm on Full-Interleaved data layout.

 On the other hand, block so that the shared memory of the GPU can be used more effectively (both the
 matrix coefficients and the right hand side of each tridiagonal system are hold on the
 shared memory of the GPU). However, computationally expensive operations, such as
 synchronizations and atomic accesses are necessary. Also this approach saturates the
 capacity of the GPU with a relatively low number of tridiagonal systems. Although the
 shared memory is much faster than the global memory, it presents some important 
 constraints to deal with. This memory is useful when the same data can be reused either by
 the same thread or by other thread of the same block of threads (CUDA block). Also,
 it is small and its use hinders the exchange
 among blocks of threads by the CUDA scheduler to overlap accesses to global memory
 with computation. Our reference implementation (the gtsvStridedBatch routine into
 the cuSPARSE package) is based of this approach, CR-PCR on Flat
 data layout (Figure \ref{fig:fig-31}).

 
 \vspace{1ex}
 \begin{figure}[htbp]
    \centering
    \includegraphics[width = 0.8\textwidth]{fig-34}
    \caption{Coarse (top) and fine (bottom) CUDA thread mapping}
    \label{fig:fig-34}
\end{figure}


\vspace{10ex}
\section{Implementation of cuHinesBatch}
As we mentioned previous in this work, we can say that Hines solver is a particular
case of Thomas solver, for this reason all of our proposed approaches are based on the
previous knowledge.

As in cuThomasBatch implementation, we are going to explore different data layouts: Flat and Full-Interleaved, explained in the previous section, and Block-Interleaved,
that similarly to the Full-Interleaved data layout, the Block-Interleaved data layout divides the set of systems in groups of systems of a given size (BS), whose elements are
stored in memory by using the strategy followed by the Full-Interleaved approach.
Same as in Figure 3.1, please note that we only illustrate one vector per system in
Figure \ref{fig:fig-35}, but in the real scenario we would have 4 vectors per Hines system.

\vspace{1ex}
\begin{figure}[htbp]
   \centering
   \includegraphics[width = 0.8\textwidth]{fig-35}
   \caption{Example of Block-Interleaved data layout with a BS equal to 2, for 4 Hines
    systems of three elements each. Point-line represe}
   \label{fig:fig-35}
\end{figure}

Although, we exploit coalescence in the memory accesses by using the Full-Interleaved
approach, the threads have to jump in memory as many elements as the number of systems to access the next element of the vector(s) (Point-lines in Figure 3.5). This could
cause an inefficient use of the memory hierarchy. This is why we study an
 additional approach, the called \textit{Block-Interleaved} data layout.  
 Using this approach we reduce thenumber of elements among consecutive
  elements of the same system, and so the jumpsin memory are not as big
   as in the previous approach (\textit{Full-Interleaved}), 
   while keeping the coalesced memory accesses.  Also, the use of the
   \textit{Block-Interleaved} data layout can take advantage better of the 
    growing importance of the bigger and bigger cachememories in the
     memory hierarchy of the current and upcoming CUDA architectures.

\subsection*{Implementation based on Shared Memory}
Unlike the previous approaches, here we explore the use of shared
 memory for our tar-get application.  
 The shared memory is much faster than the global memory; 
 however it presents some important constraints to deal with. 
  This memory is useful when thesame  data  can  be  reused  either 
   by  the  same  thread  or  by  other  thread  of  the  same block 
   of threads (CUDA block).  Also, it is small (until 48KB in the 
    architecture used) and its use hinders the exchange among blocks
    of threads by the CUDA scheduler tooverlap accesses to global memory
     with computation.

 As we can see in Algorithm 1, in our problem the elements of the
vectors $a$, $d$, $b$, $rhs$ and $p$ are reused in the \textit{Forward 
 Sweep} after computing the Backward Sweep. However, the granularity used 
 (1 thread per system) and the limit of the shared memory (48KB) prevents
  from storing all the vectors in shared memory. To be able to use 
  shared memory we have to use theBlock-Interleaveddata layout.  
  The number of systems tobe grouped (BS) is imposed by the size of
   the shared memory.  In order to address thelimitation of shared 
   memory, we only store therhsvector, as this is the vector on which 
   more accesses are carried out.  In this sense,  the more systems 
   are packed in sharedmemory, the more accesses to shared memory 
   are carried out.


\vspace{10ex}
\section{Implementation of Variable Batch}

In  this  section,  we  describe  the  techniques  used  to  deal
with  Variable  Batch,  batchof tridiagonal(Thomas and Hines) 
systems with different sizes.  Figure 3.6 graphically illustrates a 
simple example of Variable Batch composed by three vectors of 
different size.  To deal with Variable Batch, we make use of a 
widely used and extended techniquevery popular in parallel
programming, the so called padding.  This technique basically consists
of  filling  with  null  elements  those  memory  locations  between  two  different data
of  different  size.   This  is  particularly  interesting  and  
beneficial  for  GPU-basedarchitectures, where the pattern of 
access to memory is critical to achieve coalescing and reduce the 
impact of the high latency.  However, in our particular scenario, 
we haveto adapt this technique to the data-layout used, 
Full-Interleaved.  An example of this is illustrated by Figure \ref{fig:fig-36}

    \vspace{1ex}
    \begin{figure}[htbp]
        \centering
        \includegraphics[width = 0.8\textwidth]{fig-36}
        \caption{Example of the Full-Interleaved data layout for Variable Batch (Padding)}
        \label{fig:fig-36}
    \end{figure}

Regarding the CUDA thread mapping, we follow the approach based on using one
CUDA thread per tridiagonal thread (Figure 3.4-top). The reference code (gtsvStridedBatch) does not offer the possibility to compute batch of tridiagonal system of different
size. We have evaluated two different approaches, one called Computing Padding (CP)
and one called No Computing Padding (NCP). While the CP approach computes the
null elements located between different inputs, which does not affect to the final result,
the NCP only computes the no-null elements. This last can be shown as a more efficient
approach, but it is in need of a extra parameter (vector) which stores the size of the
systems and it suffers from divergence. Those threads that are in charge of computing
small systems, stop before others which have to compute large systems. This provokes
not only divergence





