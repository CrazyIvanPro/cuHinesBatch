\chapterimage{head2} % Chapter heading image

\chapter{Implementation Details}

\section{Problem Formulation}

对于一棵具有 $N$ 个结点的树 $T$,根结点编号为 0,自树根至叶的编号呈深度优先次序,
每个结点具有以下六种属性(括号内为缩写):

\begin{itemize}
    \item index(id)
    \item upper(u)
    \item lower(l)
    \item diagonal(d)
    \item right-side-hand(rhs)
    \item parent(p)  
\end{itemize}

\vspace{1ex}
其中 parent 代表的父结点的 id。

\vspace{5ex}
该问题 包含两个阶段: 

\begin{itemize}
    \item 第一阶段完成由树叶至树根的一次遍历,记作 backward sweep;
    \item 第二阶段完成由树根至叶的一次遍历,记作 forward sweep;
\end{itemize}

\vspace{1ex}
两个阶段的次序不可交换。算法伪代码如下:

\vspace{5ex}
\begin{figure}[htbp]
    \centering
    \includegraphics[width = 0.8\textwidth]{description}
    \label{fig:description}
\end{figure}


\vspace{10ex}
\section{Implementation of cuThomasBatch}
An efficient memory management is critical to achieve a good performance, but even
much more on those architectures based on a high throughput and a high memory
latency, such as the GPUs. In this sense, first we focus on presenting the different data
layouts proposed and analyze the impact of these on the overall performance. Two
different data layouts were explored: Flat and Full-Interleaved. While the Flat data
layout consists of storing all the elements of each of the systems in contiguous memory
locations, in the Full-Interleaved data layout, first, we store the first elements of each
of the systems in contiguous memory locations, after that we store the set of the second
elements, and so on until the last element.

\vspace{5ex}
\begin{figure}[htbp]
    \centering
    \includegraphics[width = 0.8\textwidth]{fig-31}
    \label{fig:fig-31}
    \caption{Example of the \textit{Flat} data layout}
\end{figure}

For sake of clarity, Figure \ref{fig:fig-31} and \ref{fig:fig-32} illustrate a simple example composed by four
different tridiagonal systems of three elements each. Please, note that we only illustrate
one vector per system in Figure 3.1, but in the real scenario we would have 4 vectors
per tridiagonal system on which are carried out the strategies above described. As
widely known, one of the most important requirements to achieve a good performance
on NVIDIA GPUs is to have contiguous threads accessing contiguous memory locations
(coalescing memory accesses). This is the main motivation behind the proposal of the
different data layouts and CUDA thread mappings. As later shown, the differences
found in the data layouts studied have important consequences on the scalability.

\begin{figure}[htbp]
    \centering
    \includegraphics[width = 0.8\textwidth]{fig-32}
    \label{fig:fig-32}
    \caption{Example of the \textit{Flat} data layout}
\end{figure}

Additionally, we have explored other data-layout, Unified-Vector. In this case, we
attempt to analyze the hierarchy of memory by exploiting the temporal locality that
there is among the different vectors. Basically, every
thread, immediately after $a_i$ is computed, has to compute also 
$b_i$ and $c_i$, in the forward step. In the backward step, the process
is similar, but in the opposite order.










